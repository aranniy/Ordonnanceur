\documentclass{article}
\usepackage[utf8]{inputenc}
\usepackage{longtable}
\usepackage{sectsty}

\title{\Large Un ordonnanceur par work stealing}
\author{Aranniy Lingeswaran et Victoria Lim}
\date{5 avril 2024}

\sectionfont{\fontsize{15}{15}\selectfont}

\begin{document}
	
	\maketitle
	
	\section*{Ordonnanceur LIFO}
	
	\subsection*{Description du travail :}
	
		TODO:
		\begin{itemize}
		\item parler des difficultés rencontrées lors de la conception, et réduire la description des fonctions...c'est peut être pas utile ?
		\item en extension, peut-être comparer les performances sur Mac et Lulu ?
		\item faire des graphes !
		\item refaire les test du Benchmark, les résultats sont étonnants mdr
	\end{itemize}
	
	
	Pour réaliser notre ordonnanceur LIFO, nous avons créé une structure de données scheduler contenant :
	
	\begin{itemize}
		\item une liste de tasks de fonctions,
		\item le nombre minimum de tâches simultanées,
		\item la position de la dernière tâche en date,
		\item le nombre de threads,
		\item le nombre de threads endormis,
		\item un pthread mutex, et
		\item un pthread cond.
	\end{itemize}
	
	La fonction \texttt{sched\_spawn} permet d’ajouter une tâche dans notre tableau tasks du scheduler. Elle agit à l’intérieur d’un mutex pour éviter l’enfilage de tâches à la même position. L’ajout de la tâche consiste à incrémenter la valeur de \texttt{task\_libre}, et de stocker la \texttt{taskfunc} et la fonction associée dans \texttt{tasks}. Après avoir enfilé la tâche, on lance un signal, à l’aide de \texttt{cond}, à un thread endormi (le choix du thread est arbitraire) pour qu’il se réveille afin qu’il effectue la tâche.
	
	La fonction \texttt{sched\_init} va allouer en mémoire la structure scheduler ainsi que ses attributs tasks et closures. On a d’abord initialisé le mutex et la variable de condition, ensuite on a stocké les valeurs données en paramètre dans le scheduler, puis on a créé un tableau de threads et chacun de ces threads va exécuter la fonction \texttt{scheduler\_thread}. On finit ensuite en attendant le retour de chaque thread à l’aide d’un \texttt{pthread\_join} et en désallouant la mémoire allouée. La fonction \texttt{scheduler\_thread} va s’exécuter dans une boucle \texttt{while(1)}, dans un mutex afin de protéger la modification des données du scheduler. S’il n’y a pas de tâches en cours, le thread courant s’endort, mais si tous les autres threads sont endormis, ils doivent les réveiller à l’aide d’un \texttt{broadcast}, libérer le lock qu’ils ont pris et sortir de la boucle. Dans ce cas-là, les threads réveillés, s’il n’y a toujours pas de tâches disponibles, vont à leur tour quitter. Toutefois, s’il y a bien une tâche, on la dépille et on l’exécute.
	
	\subsection*{Résultats :}
	
	\begin{longtable}{|c|c|c|c|c|c|}
		\hline
		\textbf{Size} & \textbf{2 cœurs} & \textbf{4 cœurs} & \textbf{6 cœurs} & \textbf{8 cœurs} & \textbf{16 cœurs} \\
		\hline
		1000 & 0.000250 & 0.000389 & 0.001632 & 0.001156 & 0.002031 \\
		10000 & 0.000849 & 0.000696 & 0.001288 & 0.001423 & 0.001543 \\
		100000 & 0.006559 & 0.004797 & 0.004718 & 0.003764 & 0.004511 \\
		1000000 & 0.071218 & 0.057436 & 0.068966 & 0.056063 & 0.037129 \\
		10000000 & 0.787602 & 0.581183 & 0.565269 & 0.564687 & 0.577924 \\
		100000000 & 8.579249 & 6.007914 & 8.397164 & 8.371198 & 4.974978 \\
		\hline
		\caption{Résultats de benchmarks en fonction du nombre de cœurs (en sec)}
	\end{longtable}
	
	\subsection*{Synthèse :}
	
	\begin{itemize}
		\item Globalement, on constate une diminution du temps d'exécution lorsque le nombre de cœurs augmente.
		\item Cette diminution n'est pas toujours linéaire et peut varier en fonction de la taille des données.
		\item Pour des tailles de données plus importantes, le temps d'exécution diminue significativement avec un nombre plus élevé de cœurs.
		\item Pour des tailles de données beaucoup moins élevé, le temps d'exécution augmente avec un nombre plus élevé de cœurs.
	\end{itemize}
	
\end{document}

